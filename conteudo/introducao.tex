%---------- Primeiro Capitulo ----------
\chapter{Introdu\c{c}\~ao}
O processo de desenvolvimento de um novo fármaco envolve diversas etapas que englobam desde a pesquisa de um determinado alvo biológico, até a descoberta de compostos com atividades biológicas desejadas e com potencial para se tornarem medicamentos a serem comercializados. Durante esse processo, inúmeras ferramentas e abordagens computacionais podem ser aplicadas visando auxiliar o pesquisador no estudo dos compostos, e também acelerar o desenvolvimento do fármaco. Nos últimos anos, devido a introdução de abordagens computacionais, principalmente nas fases inicias do processo de desenvolvimento de um fármaco, onde o foco do pesquisador é o estudo de um determinado alvo biológico,  tem difundido técnicas de desenvolvimento baseadas no ligante (Ex: Similaridade molecular, modelo farmacofórico) possibilitado assim a identificação de moléculas protótipos para ensaios biológicos \cite{rodrigues2012}. Dentre essas técnicas baseadas no ligante, um conceito já popularizado na comunidade científica é o conceito de similaridade molecular, o qual preconiza que, moléculas que possuem estruturas similares, provavelmente compartilhem propriedades físico-químicas, e atividades biológicas semelhantes \cite{singh2004reasoning}. Dessa maneira, o princípio do processo de desenvolvimento de um fármaco resume-se ao estudo de um determinado alvo biológico para o desenvolvimento de um composto ligante capaz de interagir com o alvo obtendo uma atividade biológica desejada, e em alguns casos realizar uma triagem em bancos de dados moleculares buscando por compostos similares ao ligante em questão (Virtual Screening).

Nessa perspectiva, os esforços para o desenvolvimento e sintetização de um composto ligante podem ser realizados através de sistemas computacionais que aplicam o conceito de similaridade molecular em três grandes eixos segundo \citeonline{kubinyi2008virtual}: a) Exploração computacional e bioquímica de moléculas com estruturas conhecidas (sintetizadas ou não); b) Desenvolvimento de modelos computacionais para estudo de como variações na estrutura molecular afetam a atividade molecular ou as propriedades da molécula; c) Exame de bancos de dados moleculares visando obtenção de um composto similar à estrutura do ligante projetado pelo pesquisador. 

Algumas ferramentas já tem auxiliado pesquisadores nesse sentido como por exemplo o ZINC \cite{irwin2005zinc} e PUBCHEM \cite{li2010pubchem}, ambas ferramentas web que disponibilizam bancos de dados com uma diversidade de moléculas, e também implementam algoritmos para computação de similaridade molecular, permitindo ao pesquisador realizar uma triagem em suas bases de dados a procura de um composto com determinado grau de similaridade a uma determinada estrutura molecular de interesse. Apesar dessas ferramentas auxiliarem pesquisadores a realizarem seus estudos, ainda sofrem de limitação de representação do espaço químico, onde apesar do grande número de moléculas já catalogadas em seus bancos de dados moleculares, o usuário tem seu universo de pesquisa limitado às estruturas disponíveis nesses bancos de dados. Outro problema relacionado a utilização dessas ferramentas já disponíveis é que nenhuma delas fornece ao pesquisador a possibilidade de criação de base de dados com moléculas de sua propriedade, e em alguns casos o usuário acaba disponibilizando suas estruturas em bancos colaborativos para que possam utilizar essas ferramentas para desenvolvimento de suas pesquisas, correndo riscos inclusive de perda de seus direitos autorais sobre os  seus compostos.

Neste trabalho será descrita a implementação de um algoritmo para  realização de triagem virtual baseada no conceito de similaridade molecular no sistema de banco de dados de moléculas, oriundas de fontes naturais endêmicas do bioma semiárido, denominado Natural Products Data Bank (\sigla{NatProDB}{Natural Products Data Bank}), de domínio público, para utilização em modo local (não conectado à Internet), visando: 1) Facilitar a usuários não especialistas em computação, a catalogação de moléculas e manutenção de bancos de dados moleculares, sem a necessidade de uso de bases de dados na Internet; 2) Prover mecanismos para cálculo de similaridade entre moléculas de interesse frente as moléculas depositadas no banco. A computação de similaridade implementada neste sistema é realizada através da métrica de Tversky (coeficiente de similaridade), aplicada sobre a representação computacional de moléculas através de \textit{fingerprints}. A implementação deste método é realizada por uma biblioteca livre denominada Indigo toolkit, que realiza manipulação de moléculas e sub-estruturas \cite{pavlov2011indigo}. Para os testes foi criado um banco de dados para o sistema com um conjunto de moléculas de um banco colaborativo disponibilizado pelo ZINC, e foram enxertadas nesse banco moléculas com grau de similaridade superior a 80\% a um conjunto de moléculas de entrada já testados e utilizados pela indústria farmacêutica. A avaliação do sistema foi realizada verificando os resultados obtidos pelo NatProDB, e avaliando através de matrizes de confusão a capacidade do sistema de classificar corretamente as moléculas 0.8 ou 80\% similares ou não. Os resultados destes testes, assim como os detalhes da implementação deste sistema, serão detalhados nas próximas seções deste trabalho.


                  

\section{Objetivos}

\subsection{Objetivo Geral}

Implementar algoritmo para permitir o processo de triagem virtual de moléculas baseado no conceito de similaridade molecular no sistema NatProDb

\subsection{Objetivos Espec\'ificos}

\begin{itemize}
	\item Implementar um algoritmo para cálculo de similaridade no sistema NatProDB.
	\item Avaliar a eficiência da métrica de similaridade 2D no NatProDB
\end{itemize}


