%---------- Primeiro Capitulo ----------
\chapter{Introdu\c{c}\~ao}
O processo de desenvolvimento de um novo fármaco envolve diversas etapas que englobam desde a pesquisa de um determinado alvo biológico, até a descoberta de compostos com determinadas atividades biológicas desejadas com potencial para se tornarem medicamentos a serem comercializados. Durante esse processo, inúmeras ferramentas e abordagens computacionais podem ser aplicadas visando auxiliar o pesquisador no estudo dos compostos, e também acelerar o desenvolvimento do fármaco. Nos últimos anos, devido a introdução dessas abordagens computacionais, principalmente nas fases inicias do processo de desenvolvimento de um fármaco, onde  foco do pesquisador é o estudo de um determinado alvo biológico,  tem difundido técnicas de desenvolvimento baseadas no ligante (Ex: Similaridade molecular, modelo farmacofórico) possibilitado assim a identificação de moléculas protótipos para ensaios biológicos ( RODRIGUES {\it et. al.}, 2012). Dentre essas técnicas baseadas no ligante, um conceito já popularizado na comunidade científica é o conceito de similaridade molecular, o qual preconiza que, moléculas que possuem estrutras similares, provavelmente compartilhem propriedades físico-químicas, e atividades biológicas semelhantes (SINGH, 2004). Dessa maneira, o processo de desenvolvimento de um fármaco resume-se ao estudo de um determinado alvo biológico para o desenvolvimento de um composto ligante capaz de interagir com o alvo obtendo uma atividade biológica desejada.

Nessa perspectiva, os esforços para o desenvolvimento de um ligante podem ser realizados através de sistemas computacionais que aplicam o conceito de similaridade molecular em três grandes eixos segundo (BOHM; SCHNEIDER, 2000): a) Exploração computacional e bioquímica de moléculas com estruturas conhecidas (sintetizadas ou não); b) Desenvolvimento de modelos computacionais para estudo de como variações na estrutura molecular afetam a atividade molecular ou as propriedades da molécula; c) Exame de bancos de dados moleculares visando obtenção de um composto similar à estrutura do ligante projetado pelo pesquisador.             

\section{Motiva\c{c}\~ao}

Uma das principais vantagens do uso do estilo de formata\c{c}\~ao {\ttfamily abnt-uefs.cls} para \LaTeX\ \'e a formata\c{c}\~ao \textit{autom\'atica} dos elementos que comp\~oem um documento acad\^emico, tais como capa, folha de rosto, dedicat\'oria, agradecimentos, resumo, abstract, listas de figuras, tabelas, siglas e s\'imbolos, sum\'ario, cap\'itulos, refer\^encias, etc. Outras grandes vantagens do uso do \LaTeX\ para formata\c{c}\~ao de documentos acad\^emicos dizem respeito \`a facilidade de gerenciamento de refer\^encias cruzadas e bibliogr\'aficas, al\'em da formata\c{c}\~ao~-- inclusive de equa\c{c}\~oes  matem\'aticas~-- correta e esteticamente perfeita.

\section{Objetivos}

\subsection{Objetivo Geral}

Prover um modelo de formata\c{c}\~ao \LaTeX\ que atenda \`as Normas para Elabora\c{c}\~ao de Trabalhos Acad\^emicos da UEFS e \`as Normas de Apresenta\c{c}\~ao de Trabalhos Acad\^emicos do curso de Engenharia de Computa\c{c}\~ao.

\subsection{Objetivos Espec\'ificos}

\begin{itemize}
	\item Obter documentos acad\^emicos automaticamente formatados com corre\c{c}\~ao e perfei\c{c}\~ao est\'etica.
	\item Desonerar autores da tediosa tarefa de formatar documentos acad\^emicos, permitindo sua concentra\c{c}\~ao no conte\'udo do mesmo.
	\item Desonerar orientadores e examinadores da tediosa tarefa de conferir a formata\c{c}\~ao de documentos acad\^emicos, permitindo sua concentra\c{c}\~ao no conte\'udo do mesmo.
\end{itemize}

\subsubsection{Sub-subsection}
