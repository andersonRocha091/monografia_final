\chapter{Resultados e Discussões}
\label{chap:resul}
 
 De acordo com \citeonline{Dogra2007}, a similaridade entre duas moléculas pode ser verificada quando o coeficiente de similaridade resultante da comparação de duas estruturas é superior a 80\%. Desta maneira para o primeiro teste realizado no sistema, tendo como alvo o acetaminophen e utilizando o limiar de similaridade citado anteriormente, das 50 moléculas$\geq $80\% similares (extraídas do PUBCHEM) inseridas no banco do NatProDB, o algoritmo implementado no sistema recuperou corretamente 45 delas com coeficiente de similaridade superior a 80\% (Verdadeiro Positivo $V_P$), 5 delas foram classificadas pelo NatProDB com similaridade inferior a 80\% (Falso Negativo $F_P$), como podemos verificar na tabela \ref{tab:Acetaminophen} 
 
 \begin{table}[!htb]
	\centering
	\footnotesize
	\caption[Classificação de moléculas similares ao Acetaminophen]{Classificação de moléculas similares ao Acetaminophen}
	\label{tab:Acetaminophen}	
	\begin{tabular}{p{4cm}ccc}
		\hline \SPACE
		\textbf{}&\textbf{Similaridade $\geq $ 80\%} & \textbf{Similaridade $\leq $ 80\%} \\ \hline \SPACE
	Moléculas classificadas c/ similaridade $\geq $ 80\%  &	45 & 0 \\ \hline \SPACE
	Moléculas classificadas c/ similaridade $\leq $ 80\%  & 5 & 47742\\ \hline 
	\end{tabular}
	\fonte{Pr\'oprio Autor.}
\end{table}
A partir da matriz de confusão  representada na tabela \ref{tab:Acetaminophen} podemos observar para esse teste uma Acurácia = 99,00\% equação\eqref{eq:Acuracia}, Sensibilidade = 90,00\% equação\ref{eq:Sensibilidade}, Medida F-Score = 94,70\% equação\eqref{eq:FMedida}, Precisão = 100\% equação\eqref{eq:Precisao}, Especificidade = 100\% equação\eqref{eq:Especificidade}

Para o segundo teste realizado, tendo como alvo a fluoxetina, e utilizando o mesmo limiar de similaridade aplicado anteriormente, das 50 moléculas $\geq $80\% similares a fluoxetina enxertada no NatProDB (extraídas do PUBCHEM), o algoritmo implementado no NatProDB recuperou corretamente 48 dessas estruturas ($V_P$), e apenas duas delas foram classificadas com coeficiente de similaridade inferior a 80\%($F_N$), e nenhum $F_P$ foi verificado conforme podemos observar na tabela \ref{tab:Fluoxetina}. 

 \begin{table}[!htb]
	\centering
	\footnotesize
	\caption[Resultado Obtido da Busca de Estruturas $\geq$80\% Similares a Fluoxetina]{Resultado Obtido da Busca de Estruturas $\geq$80\% Similares a Fluoxetina}
	\label{tab:Fluoxetina}	
	\begin{tabular}{p{4cm}ccc}
		\hline \SPACE
		\textbf{}&\textbf{Similaridade $\geq $ 80\%} & \textbf{Similaridade $\leq $ 80\%} \\ \hline \SPACE
	Moléculas classificadas c/ similaridade $\geq $ 80\%  &	48 & 0 \\ \hline \SPACE
	Moléculas classificadas c/ similaridade $\leq $ 80\%  & 2 & 47742\\ \hline 
	\end{tabular}
	\fonte{Pr\'oprio Autor.}
\end{table}       
A partir da tabela \ref{tab:Fluoxetina} também podemos observar que para esse teste foram obtidas: Acurácia = 99,00\% equação\eqref{eq:Acuracia}, Sensibilidade = 96,00\% equação(\ref{eq:Sensibilidade}), Medida F-Score = 97,90\% equação\eqref{eq:FMedida}, Precisão = 100\% equação\eqref{eq:Precisao},  Especificidade = 100\% equação\eqref{eq:Especificidade}.

O terceiro teste realizado teve como alvo Glyburid. Sob o mesmo limiar de similaridade dos testes anteriores, das 50 moléculas $\geq $80\% similares ao Glyburid, inseridas no banco de dados do NatProDB (extraídas do PUBCHEM), o algoritmo implementado  recuperou corretamente todas as 50 estruturas ($V_P$), e por tanto nenhuma delas foram classificadas com coeficiente de similaridade inferior a 80\%($F_N$), o que gerou uma  Acurácia = 100\%  equação\eqref{eq:Acuracia}, Sensibilidade = 100\% equação(\ref{eq:Sensibilidade}), Medida F-Score = 100\% equação\eqref{eq:FMedida}, Precisão = 100\% equação\eqref{eq:Precisao},  Especificidade = 100\% equação\eqref{eq:Especificidade}. conforme podemos observar na tabela \ref{tab:Glyburid}.

 \begin{table}[!htb]
	\centering
	\footnotesize
	\caption[Resultado Obtido da Busca de Estruturas $\geq$ 80\% Similares ao Glyburid]{Resultado Obtido da Busca de Estruturas $\geq$ 80\% Similares ao Glyburid}
	\label{tab:Glyburid}	
	\begin{tabular}{p{4cm}ccc}
		\hline \SPACE
		\textbf{}&\textbf{Similaridade $\geq $ 80\%} & \textbf{Similaridade $\leq $ 0.8} \\ \hline \SPACE
	Moléculas classificadas c/ similaridade $\geq $ 0.8  &	50 & 0 \\ \hline \SPACE
	Moléculas classificadas c/ similaridade $\leq $ 0.8  & 0 & 47742\\ \hline 
	\end{tabular}
	\fonte{Pr\'oprio Autor.}
\end{table} 

O quarto teste realizado teve como alvo a molécula Imatinib. A triagem de moléculas no banco do NatProDB cujo coeficiente de similaridade $\geq$ 80\% com relação a molécula-alvo, retornou 49 das 50 estruturas $\geq$80\% similares ao imatinib enxertadas no banco. Dessa maneira foram obtidos 49  $V_P$, apenas 1 $F_P$, e novamente nenhum $F_N$ foi verificado. A partir desse resultado foram obtidas: Acurácia = 99,00 equação\eqref{eq:Acuracia}, Sensibilidade = 98,00\% equação(\ref{eq:Sensibilidade}), Medida F-Score = 98,90\% \eqref{eq:FMedida}, Precisão = 100,00\% equação\eqref{eq:Precisao},  Especificidade = 100,00\% \eqref{eq:Especificidade}. conforme podemos observar na tabela \ref{tab:Imatinib}

 \begin{table}[!htb]
	\centering
	\footnotesize
	\caption[Resultado Obtido da Busca de Estruturas $\geq$80\% Similares ao Imatinib]{Resultado Obtido da Busca de Estruturas $\geq$80\% Similares ao Imatinib}
	\label{tab:Imatinib}	
	\begin{tabular}{p{4cm}ccc}
		\hline \SPACE
		\textbf{}&\textbf{Similaridade $\geq$80\%} & \textbf{Similaridade $\leq $80\%} \\ \hline \SPACE
	Moléculas classificadas c/ similaridade $\geq $ 80\%  &	50 & 0 \\ \hline \SPACE
	Moléculas classificadas c/ similaridade $\leq $ 80\%  & 0 & 47742\\ \hline 
	\end{tabular}
	\fonte{Pr\'oprio Autor.}
\end{table}

O quinto teste realizado no sistema teve como alvo o Isosorbid. A consulta realizada no banco do NatProDB por estruturas 80\% similares à molécula-alvo, retornou todas as 34 moléculas $\geq$ 80\% similares as que foram enxertadas no banco do sistema, não gerando assim nenhum falso-positivo ou falso-negativo. Consequentemente foram obtidas a partir dos resultados apresentados na tabela \ref{tab:Isosorbid}: Acurácia = 100\% equação\eqref{eq:Acuracia}, Sensibilidade = 100,00\% equação(\ref{eq:Sensibilidade}), Medida F-Score = 100,00\% \eqref{eq:FMedida}, Precisão = 100,00\% equação\eqref{eq:Precisao},  Especificidade = 100,00\% equação\eqref{eq:Especificidade}.

 \begin{table}[!htb]
	\centering
	\footnotesize
	\caption[Resultado Obtido da Busca de Estruturas $\geq$80\%  Similares ao Isosorbid]{Resultado Obtido da Busca de Estruturas $\geq$80\% Similares ao Isosorbid}
	\label{tab:Isosorbid}	
	\begin{tabular}{p{4cm}ccc}
		\hline \SPACE
		\textbf{}&\textbf{Similaridade $\geq $80\%} & \textbf{Similaridade $\leq $80\%} \\ \hline \SPACE
	Moléculas classificadas c/ similaridade $\geq $80\%  &	34 & 0 \\ \hline \SPACE
	Moléculas classificadas c/ similaridade $\leq $80\%  & 0 & 47758\\ \hline 
	\end{tabular}
	\fonte{Pr\'oprio Autor.}
\end{table}

Para o sexto teste foi selecionada como molécula-alvo o Vinblastine. A triagem realizada no banco de dados do NatProDB por estruturas $\geq$ 80\% similares a molécula-alvo, recuperou todas as 50 moléculas $\geq$80\% similares ao Vinblastine enxertadas no banco de dados do sistema. Novamente, assim como no teste anterior, não foi verificada a ocorrência de falsos-positivos e/ou falsos-negativos, resultando em medidas de: Acurácia = 100,00\% equação\eqref{eq:Acuracia}, Sensibilidade = 100,00\% equação(\ref{eq:Sensibilidade}), Medida F-Score = 100,00\% equação\eqref{eq:FMedida}, Precisão = 100,00\% equação\eqref{eq:Precisao},  Especificidade = 100,00\% equação\eqref{eq:Especificidade}. Conforme podemos verificar pela tabela \ref{tab:Vinblastine}

 \begin{table}[!htb]
	\centering
	\footnotesize
	\caption[Resultado Obtido da Busca de Estruturas $\geq$ 80\% Similares ao Vinblastine]{Resultado Obtido da Busca de Estruturas $\geq$80\% Similares ao Vinblastine}
	\label{tab:Vinblastine}	
	\begin{tabular}{p{4cm}ccc}
		\hline \SPACE
		\textbf{}&\textbf{Similaridade $\geq $ 80\%} & \textbf{Similaridade $\leq $ 80\%} \\ \hline \SPACE
	Moléculas classificadas c/ similaridade $\geq $ 80\%  &	50 & 0 \\ \hline \SPACE
	Moléculas classificadas c/ similaridade $\leq $ 80\%  & 0 & 47742\\ \hline 
	\end{tabular}
	\fonte{Pr\'oprio Autor.}
\end{table}

\newpage
O sétimo teste realizado teve como molécula-alvo o Propanolol. A consulta realizada no NatProDB buscando as estruturas $\geq$ 80\% similares ao Propanolol recuperou 42 das 50 moléculas enxertadas no banco do sistema (Verdeiros positivos $V_P$), e as 8 moléculas restantes foram classificadas com coeficiente de similaridade inferior a 0.8 (Falso negativos $F_P$). A partir desse dados foram obtidas a partir da tabela (\ref{tab:Propanolol}): Acurácia = 99,00\% equação\eqref{eq:Acuracia}, Sensibilidade = 84,00\%  equação(\ref{eq:Sensibilidade}), Medida F-Score = 91,00\% equação\eqref{eq:FMedida}, Precisão = 100,00\% equação\eqref{eq:Precisao},  Especificidade = 100,00\% equação\eqref{eq:Especificidade}.  

Os resultados desse teste podem ser verificado na matriz de confusão (\ref{tab:Propanolol})
\begin{table}[!htb]
	\centering
	\footnotesize
	\caption[Resultado Obtido da Busca de Estruturas $\geq$80\% Similares ao Propanolol]{Resultado Obtido da Busca de Estruturas $\geq$80\% Similares ao Propanolol}
	\label{tab:Propanolol}	
	\begin{tabular}{p{4cm}ccc}
		\hline \SPACE
		\textbf{}&\textbf{Similaridade $\geq $ 80\%} & \textbf{Similaridade $\leq $ 80\%} \\ \hline \SPACE
	Moléculas classificadas c/ similaridade $\geq $ 80\%  &	42 & 0 \\ \hline \SPACE
	Moléculas classificadas c/ similaridade $\leq $ 80\%  & 8 & 47742\\ \hline 
	\end{tabular}
	\fonte{Pr\'oprio Autor.}
\end{table}

\begin{table}[!htb]
	\centering
	\footnotesize
	\caption[Resumo dos Resultados Obtidos nos Testes Realizados]{Resumo dos Resultados Obtidos nos Testes Realizados}
	\label{tab:ResumoResultados}	
	\begin{tabular}{|c|c|c|c|c|c|c|}
		\hline \SPACE
		\textbf{Teste}&\textbf{Acurácia}&\textbf{Sensibilidade \%}&\textbf{F-Score \%}&\textbf{Precisão \%}&\textbf{Especificidade \%}&\textbf{FPR \%} \\ \hline \SPACE
	1\textordmasculine &99,00&90,00&94,70&100,00&100,00&0,00 \\ \hline \SPACE
	2\textordmasculine &99,00&96,00&97,90&100,00&100,00&0,00\\ \hline 
	3\textordmasculine &100,00&100,00&100,00&100,00&100,00&0,00\\ \hline
	4\textordmasculine &99,00&98,00&98,90&100,00&100,00&0,00\\ \hline
	5\textordmasculine &100,00&100,00&100,00&100,00&100,00&0,00\\ \hline
	6\textordmasculine &100,00&100,00&100,00&100,00&100,00&0,00\\ \hline
	7\textordmasculine &99,00&84,00&91,00&100,00&100,00&0,00\\ \hline
	\end{tabular}
	\fonte{Pr\'oprio Autor.}
\end{table}
 
De acordo com \citeonline{davis2006relationship}, embora as medidas de acurácia e F-medida forneçam uma medida global da qualidade da classificação, na qual quão mais próximos de 100,00\% os valores obtidos para as mesmas melhor a classificação, o autor relata que uma classificação ideal ocorre quando a Sensibilidade \eqref{eq:Sensibilidade} $\simeq 100,00$\% , e FPR \eqref{eq:FPR}  $\simeq 0$\%. Sob essa perspectiva, tomando como base o pior caso dos testes realizados no sistema, que teve como alvo a molécula de propranolol - que reportou $F_N$ = 8, podemos facilmente identificar a partir tabela de resumo dos testes (\ref{tab:ResumoResultados}) que foram obtidas uma Sensibilidade = 84\%, e FPR = 0\%, o que indica que o método aplicado para separar as moléculas $\geq$ 80\% similares ao propranolol obteve uma resposta satisfatória, uma vez que ambas sensibilidade e FPR se aproximaram respectivamente de 100\% e 0\% (classificador ideal). Estendendo tal verificação para todos os testes realizados, foi observado que não ocorreram $F_P$ nos testes realizados, ou em outras palavras, o método  implementado para triagem virtual de moléculas baseado no conceito 80\% de  similaridade não retornou nenhuma molécula com coeficiente de similaridade inferior a  para cada uma das moléculas-alvo testadas. Consequentemente, as sensibilidades verificadas em todos os testes realizados variam dentro de um intervalo de [84,00\% - 99,00\%], o que de acordo com \citeonline{davis2006relationship}  pode ser considerado como um comportamento próximo de um classificador ideal.




            


   

   