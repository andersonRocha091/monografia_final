\begin{abstract}
The developing process of a medicine has many stages. Since the project
and study of a possible pharmacophore compound until its synthesizing. Most efforts on the
development of a new medicine are aided by computers. The
computational tools allow the researchers both cataloguing in a more efficient way their study
structures and evaluate in a faster way biochemical, structural features, and the
molecules behavior during structural variations.Moreover, some of these tools also helps in the search for similar structures, that is a key concept in which most efforts towards to develop a new medicine are based.
This work describes the adaptation of appropriate algorithms for computing molecular
similarities in a software developed for molecules tabulation in a database enabling the researcher to screen similar structures to the ones that are in his study concerns at his own database. A differential in this work lies in the fact that this tool is made freely available to any user, and due the inner characteristics of its implementation, face to others tools that exist with the same purpose, this software does not need much hardware resources to run, in other words, it is compatible with common computers, and any operational system used by the researcher. In resume, the system works according to these steps: once the molecules are registered in the user database, the researcher is able to verify in his base if there are structures with a desired degree of similarity to the molecule queried. So, the system converts the inserted molecule, and those ones in the database in fingerprints format in order to use the Tversky's metrics obtaining a table of similarity, and finally retrieving to the researcher all molecules in his database whose similarity coefficient  it is equal or superior to the one desired (regarded to the queried molecule). 
\end{abstract}
