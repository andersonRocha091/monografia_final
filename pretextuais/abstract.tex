\begin{abstract}
The developing process of a medicine is made of different stages, that come from the project
and study of a possible pharmacophore compound until its synthesizing. Most efforts on the
development of a new medicine are made in assisted way troughthrough computers. The
computational tools allow the researchers both cataloguing in a more efficient way their study
structures and evaluate in a faster way biochemical, structural characteristics, and the
molecules behavior during structural variations. Besides this, the tools also help in the search for similar structures, that is a concept in most efforts in a new medicine project are based.
This work describes the adaptation of appropriate algorithms for computing molecular
similarities in a software developed for the tabulation of molecules in a database, aiming to
allow the researcher to browse, in his own database, similar structures to the ones that are in his study concerns. A differential in this work is that this tool will be made freely available to any user, and due the inner characteristics of its implementation, face to other tools that exist with the same purpose, this software will not need much hardware resources for its execution, in other words, it is compatible with common computers, and also with any operational system used by the researcher. Briefly, the system will work in such manner: once the molecules are registered in the user database, the researcher will be able to verify in his base if there are structures with a desired degree of similarity to the molecule inserted for consultation. The system converts the inserted molecule, and the presents in the database, in fingerprints for the application of Tanimoto's metric, to obtain a table of similarity, returning finally to the researcher all the molecules in his database that have a degree of similarity equal or superior to the one desired (relating to the entrance molecule). 
\end{abstract}
