\begin{resumo}
O processo de desenvolvimento de um fármaco é realizado em diversas etapas, que vão desde
o projeto e estudo de um possível composto farmacofórico até sua sintetização. Grande parte
dos esforços empregados para o desenvolvimento de um novo medicamento são realizados de
forma assistida em computadores. As ferramentas computacionais permitem aos pesquisadores
tanto catalogar de forma mais eficiente suas estruturas de estudo, quanto avaliar de forma mais rápida características estruturais, bioquímicas, e o comportamento de moléculas ao sofrer variações estruturais. Além disso, algumas dessas ferramentas também realizam triagem em bancos de dados moleculares baseados no conceito de similaridade molecular, que é um conceito chave no qual se baseiam grande parte dos esforços dirigidos ao projeto de um novo fármaco. O presente trabalho descreve a adaptação de um algoritmo apropriado para
computação de similaridade molecular em um software desenvolvido para catalogação de
moléculas em banco de dados, visando possibilitar ao pesquisador procurar em sua própria
base de dados, estruturas similares a moléculas de seu interesse de estudo. Um diferencial deste trabalho é que essa ferramenta será disponibilizada livremente para qualquer usuário, e devido às características inerentes à sua implementação, em contrapartida à outras ferramentas
disponíveis para o mesmo fim, este software não necessitará de muitos recursos de hardware
para sua execução, ou seja, é compatível com computadores comuns, e também com qualquer
sistema operacional utilizado pelo pesquisador. De forma resumida, o sistema se comporta da
seguinte forma: Uma vez cadastradas as moléculas e seus respectivos SMILES no banco de dados do usuário, o
pesquisador poderá consultar em seu banco se há ocorrência de estruturas com um grau de
similaridade desejado à molécula inserida para consulta. O sistema converte os SMILES da molécula de
entrada, e das presentes no banco em fingerprints para aplicação da métrica de Tversky para
obtenção de um coeficiente de similaridade, retornando por fim para o pesquisador todas as
moléculas do seu banco que possuem um grau de similaridade igual ou superior ao desejado
(em relação à molécula de entrada). O algoritmo adotado para computação de similaridade molecular foi testado através de uma adaptação do conceito de matriz de confusão e teve como entrada 7 moléculas já disponibilizadas como fármacos, e para os testes realizados no sistema foram verificados que a classificação de moléculas com coeficiente de similaridades $\geq$ 0,8 no sistema se aproximou do comportamento de uma classificação ideal.
\end{resumo}	
