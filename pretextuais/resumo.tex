\begin{resumo}
O processo de desenvolvimento de um fármaco é realizado em diversas etapas, que vão desde
o projeto e estudo de um possível composto farmacofórico até sua sintetização. Grande parte
dos esforços empregados para o desenvolvimento de um novo medicamento são realizados de
forma assistida em computadores. As ferramentas computacionais permitem aos pesquisadores
tanto catalogar de forma mais eficiente suas estruturas de estudo, quanto avaliar de forma maisrápida características estruturais, bioquímicas, e o comportamento de moléculas ao sofrer
variações estruturais. Além disso, essas ferramentas também auxiliam na busca por estruturas
similares, que é um conceito em que se baseia grande parte dos esforços dirigidos ao projeto de um novo fármaco. O presente trabalho descreve a adaptação de algoritmos apropriados para
computação de similaridade molecular em um software desenvolvido para catalogação de
moléculas em banco de dados, visando possibilitar ao pesquisador procurar em sua própria
base de dados, estruturas similares às de seu interesse de estudo. Um diferencial deste trabalhoé que essa ferramenta será disponibilizada livremente para qualquer usuário, e devido às características inerentes à sua implementação, em contrapartida à outras ferramentas
disponíveis para o mesmo fim, este software não necessitará de muitos recursos de hardware
para sua execução, ou seja, é compatível com computadores comuns, e também com qualquer
sistema operacional utilizado pelo pesquisador. De forma resumida, o sistema se comportará da
seguinte forma: Uma vez cadastradas as moléculas no banco de dados do usuário, o
pesquisador poderá consultar em seu banco se há ocorrência de estruturas com um grau de
similaridade desejado à molécula inserida para consulta. O sistema converte a molécula de
entrada, e as presentes no banco em fingerprints para aplicação da métrica de Tanimoto para
obtenção de um índice de similaridade, retornando por fim para o pesquisador todas as
moléculas do seu banco que possuem um grau de similaridade igual ou superior ao desejado
(em relação à molécula de entrada).
\end{resumo}	
